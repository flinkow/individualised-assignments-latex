\documentclass[answers, a4paper, 10pt]{exam}

\usepackage{amsmath}
\usepackage{amssymb}

\usepackage{tikz}
\usepackage{pgfplots}
\pgfplotsset{compat=newest}
\usepgfplotslibrary{fillbetween}

\usepackage{luacode}
\usepackage{luapackageloader}

\begin{luacode*}
    package.path = package.path .. ";../../random/?.lua;../../wsymcomp/?.lua"
    package.cpath = package.cpath .. ";../../symcomp/bin/?.so"
\end{luacode*}

% THIS IS WHERE SYMBOLIC COMPUTATION & RANDOMIZATION HAPPENS
\begin{luacode*}
    symcomp = require "wsymcomp"
    random = require "random"

    local factory = 
    {
        { "a", random.integer() },
        { "b", random.integer() }
    }

    random.create("a*x**3+b*x**2", factory)

    -- if f(x)=a*x^3+b*x^2 we can ensure that
    -- f has zero x=0 (multiplicity 2)
    -- f has zero x=b/a (multiplicity 1)
    f = symcomp.expr(random.create("a*x**3+b*x**2", factory))

    d1 = symcomp.diff(f, "x")
    d2 = symcomp.diff(d1, "x")
    d3 = symcomp.diff(d2, "x")

    -- TODO: zeros, min, max, inflections?
    fzeros = symcomp.solve(f, "x")
\end{luacode*}

% THESE WILL BE REFACTORED INTO AN OWN LATEX PACKAGE
\newcommand{\scprint}[1]{\luaexec{tex.print{symcomp.getstringlatex(#1)}}}
\newcommand{\print}[1]{\luaexec{tex.print{#1}}}
\newcommand{\printintegral}[4]{\luaexec{tex.print{symcomp.printintegral(#1, #2, #3, #4)}}}

\begin{document}

\begin{questions}

% BASED ON DIFFERENTIALRECHNUNG AUFGABE 12. ON PAGE 5
% \footnote{Arbeitsheft zur Vorlesung Ingenieurmathematik -- Differentialrechnung: Aufgabe 12 \textbullet, S. 5}
\question[k]
Given the function
\begin{equation*}
    f(x)=\scprint{f}
\end{equation*}
\begin{parts}
\part
Sketch $f, f'$ and $f''$ in one coordinate system.

\part Identify all of the minimum and maximum points and find its inflection points.
\end{parts}
\begin{solution}
\begin{parts}
\part
First, calculate the derivatives
\begin{align*}
    f(x)&=\scprint{f}\\
    f'(x)&=\scprint{d1}\\
    f''(x)&=\scprint{d2}\\
    f'''(x)&=\scprint{d3}
\end{align*}
%
\begin{center}
\begin{tikzpicture}
\begin{axis}[
    axis lines = center,
    xlabel = $x$,
    ylabel = $y$,
    %ymin = -10,
    %ymax = 10,
    %xmin = -10,
    %xmax = 10,
    enlargelimits,
    samples = 200,
    enlargelimits,
    no markers,
    legend pos=outer north east
]

% this code is moved to Lua because \addplot does not expand its arguments
\begin{luacode*}
    tex.print("\\addplot{" .. symcomp.getstringpgf(f) .. "};")
    tex.print("\\addlegendentry{$f(x)$}")
    tex.print("\\addplot{" .. symcomp.getstringpgf(d1) .. "};")
    tex.print("\\addlegendentry{$f'(x)$}")
    tex.print("\\addplot{" .. symcomp.getstringpgf(d2) .. "};")
    tex.print("\\addlegendentry{$f''(x)$}")
\end{luacode*}

\end{axis}
\end{tikzpicture}
\end{center}

\part
%The function $f$ has the double zero $x_0=0$ and the single zero $x_1=\frac{3}{2}$.
\scprint{symcomp.printzeros(f, "x")}
%The first derivative has zeros $x_2=0$ and $x_3=1$.
\scprint{symcomp.printzeros(d1, "x", "$f'$")}

\scprint{symcomp.printminmax(f, "x")}

Because $f''(x_2)>0$ there is a maximum at $(0, 0)$ and because of $f''(x_3)>0$ there is a minimum at $(1, -1)$.

The second derivative has a zero $x_4=\frac{1}{2}$. Because $f'''(x_4)\neq 0$ there is an inflection point at $(\frac{1}{2},-\frac{1}{2})$.
\end{parts}
\end{solution}

\end{questions}

\end{document}