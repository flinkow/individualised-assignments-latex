\documentclass[answers, a4paper, 10pt]{exam}

\usepackage{amsmath}
\usepackage{amssymb}

\usepackage{tikz}
\usepackage{pgfplots}
\pgfplotsset{compat=newest}
\usepgfplotslibrary{fillbetween}

\usepackage{luacode}
\usepackage{luapackageloader}

\begin{luacode*}
    package.path = package.path .. ";../random/?.lua;"
    package.cpath = package.cpath .. ";../symcomp/bin/?.so"
\end{luacode*}

% THIS IS WHERE SYMBOLIC COMPUTATION & RANDOMIZATION HAPPENS
\begin{luacode}
    symcomp = require "wsymcomp"
    random = require "random"

    -- choose one of the combinations of { f, g }
    possible = { { "x**2", "-2*x+8" }, { "1/2*x**2", "x+4" }, { "2*x**2", "2*x+4" } }
    choice = random.oneof(possible)

    f = symcomp.expr(choice[1])
    g = symcomp.expr(choice[2])

    -- or:
    -- f = symcomp.expr(random.parabola())
    -- g = symcomp.expr(random.slope())

    points = symcomp.findintersections(f, g, "x")

    assert(#points == 2, "expected two intersection points")

    -- get the points of intersections, where p1.x < p2.x
    p1 = points[1]
    p2 = points[2]

    -- subtract f(x) from g(x)
    gsubf = symcomp.sub(g, f)

    -- integrate g(x)-f(x) between the intersection points
    area = symcomp.integrate(gsubf, p1, p2, "x")
\end{luacode}

% THESE WILL BE REFACTORED INTO AN OWN LATEX PACKAGE
\newcommand{\scprint}[1]{\luaexec{tex.print{symcomp.getstringlatex(#1)}}}
\newcommand{\print}[1]{\luaexec{tex.print{#1}}}
\newcommand{\printintegral}[4]{\luaexec{tex.print{symcomp.printintegral(#1, #2, #3, #4)}}}

\begin{document}

\begin{questions}

% BASED ON INTEGRALRECHNUNG AUFGABE 2. ON PAGE 5
\question[k]
How big is the parabolic segment between the parabola $f(x)=\scprint{f}$ and the line $g(x)=\scprint{g}$?

Sketch a graph to visualize the desired area.
\begin{solution}
% This is an example of a reusable text snippet: it will print the points where the functions intersect.
\print{symcomp.printintersections(f, g, "x")}
Thus, the area is
\begin{equation*}
    A=\printintegral{"g(x)-f(x)"}{p1}{p2}{"x"}=
      \printintegral{gsubf}{p1}{p2}{"x"}=
      \left[\scprint{gsubf}\right]_{\scprint{p1}}^{\scprint{p2}}=
      \scprint{area}
\end{equation*}
%
\begin{center}
\begin{tikzpicture}
\begin{axis}[
    axis lines = center,
    xlabel = $x$,
    ylabel = $y$,
    enlargelimits,
    samples = 100,
    enlargelimits,
    no markers,
    legend pos=outer north east
]

% this code is moved to Lua because \addplot does not expand its arguments
\begin{luacode*}
    tex.print("\\addplot[name path=P] {" .. symcomp.getstringpgf(f) .. "};")
    tex.print("\\addlegendentry{$f(x)=" .. symcomp.getstringlatex(f) .. "$}")
    tex.print("\\addplot[name path=L, dashed] {" .. symcomp.getstringpgf(g) .. "};")
    tex.print("\\addlegendentry{$g(x)=" .. symcomp.getstringlatex(g) .. "$}")

    tex.print("\\addplot[gray, fill opacity=0.25] fill between[of=P and L, soft clip={domain=" .. symcomp.getstringpgf(p1) .. ":" .. symcomp.getstringpgf(p2) .. "}];")
\end{luacode*}

\end{axis}
\end{tikzpicture}
\end{center}
\end{solution}

\end{questions}

\end{document}